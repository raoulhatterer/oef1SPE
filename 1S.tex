% Created 2017-07-11 mar. 01:10
\documentclass[11pt]{article}
\usepackage[utf8]{inputenc}
\usepackage[T1]{fontenc}
\usepackage{fixltx2e}
\usepackage{graphicx}
\usepackage{grffile}
\usepackage{longtable}
\usepackage{wrapfig}
\usepackage{rotating}
\usepackage[normalem]{ulem}
\usepackage{amsmath}
\usepackage{textcomp}
\usepackage{amssymb}
\usepackage{capt-of}
\usepackage{hyperref}
\author{hatterer}
\date{\today}
\title{}
\hypersetup{
 pdfauthor={hatterer},
 pdftitle={},
 pdfkeywords={},
 pdfsubject={},
 pdfcreator={Emacs 24.5.1 (Org mode 8.3.5)}, 
 pdflang={English}}
\begin{document}

\tableofcontents

\section{7p74 Distinguer un pigment d'un colorant}
\label{sec:orgheadline3}
\subsection{énoncé}
\label{sec:orgheadline1}
Dans quatre tubes à essais contenant 2mL d'eau, on introduit respectivement:
une pointe de spatule d'hélianthine (a),
de charbon (b),
d'indigo (c),
et de vert malachite (d).
Après agitation, on observe le contenu des tubes.
\subsection{Question}
\label{sec:orgheadline2}
Quel sont les pigments et les colorants parmis les matières colorées introduites?

\section{12p75 Ecrire une formule topologique}
\label{sec:orgheadline5}
\subsection{énoncé}
\label{sec:orgheadline4}
Donner la formule topologique des molécules ayant la formule semi-dévelopée suivante:

\section{15p75 Interpréter la structure d'une molécule}
\label{sec:orgheadline8}
\subsection{Énoncé}
\label{sec:orgheadline6}
La canthaxanthine est un pigment qui donne au plumage des oiseaux une couleur rouge.
C'est un additif alimenataire autorisé pour les animaux.
\subsection{Question}
\label{sec:orgheadline7}
Toutes les doubles liaisons sont-elles conjuguées?
\section{8p74 Etudier l'influence d'un paramètre sur la couleur d'une espèce}
\label{sec:orgheadline11}
\subsection{Énoncé}
\label{sec:orgheadline9}
Les baies de troène sont riches en anthocyanes.
Jadis, les parties rouges et violettes des cartes à jouer étaient colorées par le jus des baies de troène, additionné de sulfate acide de potassium pour le rouge, d'urine pour le violet et de potasse (obtenu par macération des cendres dans l'eau) pour les bleus.

La valeur du pH d'une solution de sulfate acide de potassium est inférieure à 7.
L'urine a un pH compris entre 7 et 7,5.
Celui d'une solution de potasse est très supérieur à 7.

\subsection{Question}
\label{sec:orgheadline10}
\begin{enumerate}
\item Quel paramètre influence la couleur de l'anthocyane ?
\item Comment nomme-t-on de telles espèces?
\end{enumerate}

\section{10p74 Interpréter la couleur d'un colorant}
\label{sec:orgheadline14}
\subsection{Énoncé}
\label{sec:orgheadline12}
On réalise la chromatographie du colorant vert d'un sirop de menthe. 
Sur une plaque CCM , on dépose la tantrazine jaune (J), le bleu patenté (B) (dont la couleur est proche du cyan) et le colorant vert (V).
\subsection{Questions}
\label{sec:orgheadline13}
\begin{enumerate}
\item Que signifie l'acronyme CCM?
\item Interpréter la couleur du colorant vert à l'aide du cercle chromatique simplifé.
\end{enumerate}
\end{document}